\input{../template.tex}

% Dokumentinformationen
\newcommand{\SUBJECT}{Report}
\newcommand{\TITLE}{Cloud Infrastructre Lab 4}

\begin{document}
	
% Front page
\title{\TITLE}
\subject{\SUBJECT}
\author{\SECONDAUTHOR}
\author{\AUTHOR}
\affil{\INSTITUTE}
\date{\today}
\maketitle

% Table of contents
\tableofcontents


%  03.11.2016, 23:55, as PDF to beat.stettler@ins.hsr.ch. 

% Describe how Qemu KVM and Docker work (technical explanation). Also make sure you highlight the 
% differences, pro’s and con’s. 

% Setup and configuration of the Qemu-KVM environment (with explanations) 

% Qemu network configuration 

% A short manual of how to get the ELK stack up and running in Docker 

% Docker network configuration 

% Docker Compose file for the ELK stack 

% Qemu-KVM and Docker networking configuration files/scripts 

\section{Qemu KVM}
\subsection{Grundlegendes}

\subsection{Beschreibung}
\paragraph{Vorteile}
\begin{itemize}
	\item 
\end{itemize}
\paragraph{Nachteile}
\begin{itemize}
	\item 
\end{itemize}

\subsection{Setup}
\subsubsection{Abhängigkeitn}
\begin{enumerate}
	\item \lstinline|sudo dnf install VirtualBox vagrant vagrant-libvirt|
	\item \lstinline|sudo dnf install @virtualization docker-io libvirt-docs libvirt-wireshark|
	\item \lstinline|sudo service libvirtd start|
\end{enumerate}

\subsubsection{Image herunterladen}
\begin{enumerate}
	\item Cirros Festplatten Image herunterladen: \url{http://download.cirros-cloud.net/0.3.4/cirros-0.3.4-x86_64-disk.img}
	\item Festplatten Image nach \lstinline|/var/lib/libvirt/images/| verschieben
	\item Virtmanager starten: \lstinline|sudo virt-manager|
\end{enumerate}

\subsection{Network Configuration}


\section{Docker}
\subsection{Grundlegendes}

\subsection{Beschreibung}
\paragraph{Vorteile}
\begin{itemize}
	\item 
\end{itemize}
\paragraph{Nachteile}
\begin{itemize}
	\item 
\end{itemize}

\subsection{ELK Stack Setup}

\subsection{Network Configuration}

\subsection{Docker Compose File ELK Stack}




\section{Vergleich der Technologien}


\appendix

\section{Schemes}
% TODO physical topology (map of the network, datacenter outlet)
% TODO logical topology (map of the network, building block view, layer2 map(vlan), layer 3 map (ospf, areas??) )

\subsection{Datacenter Design}
Das Schema ist im Anhang \ref{appendix:datacenter} zu finden.

\subsection{3 Tier des Application Layers}
Das Schema ist im Anhang \ref{appendix:3_tier_application_layer_physical} zu finden.



% List of tables
\listoftables


\label{appendix:datacenter}
\includepdf[pages={1},landscape=true]{appendix/schemes/datacenter.pdf}

\label{appendix:3_tier_application_layer_physical}
\includepdf[pages={1}]{appendix/schemes/3_tier_application.pdf}


% Code Listings
% \lstlistoflistings

% List of figures
% \listoffigures

% Bibliography
% \bibliographystyle{plain} 
% \bibliography{literatur}


\end{document}

