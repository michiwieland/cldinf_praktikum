\input{../template.tex}

% Dokumentinformationen
\newcommand{\SUBJECT}{Report}
\newcommand{\TITLE}{Cloud Infrastructre Lab 2}

\begin{document}
	
% Front page
\title{\TITLE}
\subject{\SUBJECT}
\author{\SECONDAUTHOR}
\author{\AUTHOR}
\affil{\INSTITUTE}
\date{\today}
\maketitle

% Table of contents
\tableofcontents


\section{Aufgabenstellung}
\subsection{Einleitung}
Gemäss der Aufgabenstellung

\begin{quotation}
	The fictional company BetaHouse Inc. is specialized in the high-frequency trading sector. In this exercise you will design the LAN part of the network, based on customer specifications. Part two will focus on the WAN integration with several remote locations throughout Switzerland (10+).
	
	Start with analyzing and characterizing the desired goals. Then go ahead with a logical network design. Based on that define the physical network. Document, review and change if needed until you get a satisfying solution. 
	
	\textbf{It’s important to describe your design decisions in your documentation.}
\end{quotation}

\subsection{Angaben zur Firma}

\begin{description}
	\item[Firmenname] BetaHouse Inc.
	\item[Niederlassungen] 10+, verbunden über WAN
\end{description}

\subsection{Vorgehen}

\begin{enumerate}
	\item Analyse der Kundenansprüche
	\item Aufgbau des Campus LAN Design und Addressing Scheme
	\item 
\end{enumerate}

\section{LAN Design Headquarter}

\subsection{Kundenansprüche an das Netzwerk}

\subsection{Campus LAN Design}

\subsubsection{Logische Netzwerkkarte}
%MUST cover WAN access
%MUST cover Internet access
\subsubsection{Physische Netzwerkkarte}

\subsubsection{Adressierungs- \& Namensschema}

\subsection{Technologie \& Protokolle}
\subsubsection{Layer 2}
\subsubsection{Layer 3}

\subsection{Benötigte Komponenten}

\section{WAN Design}

\subsection{Kundenansprüche an das Netzwerk}

\subsection{WAN Network Design}
%MUST cover remote locations, external services like web, mail
%MUST include security considerations

\subsubsection{Adressierungs- \& Namensschema}

\subsubsection{Logische Netzwerkkarte}
%MUST cover HQ and 3 Branch Offices
\subsubsection{Physische Netzwerkkarte}

\subsection{Technologie \& Protokolle}
\subsubsection{Layer 2}
\subsubsection{Layer 3}

\subsection{Benötigte Komponenten}

\appendix

\section{Schemes}
% TODO physical topology (map of the network, datacenter outlet)
% TODO logical topology (map of the network, building block view, layer2 map(vlan), layer 3 map (ospf, areas??) )

\subsection{Datacenter Design}
Das Schema ist im Anhang \ref{appendix:datacenter} zu finden.

\subsection{3 Tier des Application Layers}
Das Schema ist im Anhang \ref{appendix:3_tier_application_layer_physical} zu finden.



% List of tables
\listoftables


\label{appendix:datacenter}
\includepdf[pages={1},landscape=true]{appendix/schemes/datacenter.pdf}

\label{appendix:3_tier_application_layer_physical}
\includepdf[pages={1}]{appendix/schemes/3_tier_application.pdf}


% Code Listings
% \lstlistoflistings

% List of figures
% \listoffigures

% Bibliography
% \bibliographystyle{plain} 
% \bibliography{literatur}


\end{document}

