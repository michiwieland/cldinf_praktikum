\input{../template.tex}

% Dokumentinformationen
\newcommand{\SUBJECT}{Report}
\newcommand{\TITLE}{Cloud Infrastructre Lab 2}

\begin{document}
	
% Front page
\title{\TITLE}
\subject{\SUBJECT}
\author{\SECONDAUTHOR}
\author{\AUTHOR}
\affil{\INSTITUTE}
\date{\today}
\maketitle

% Table of contents
\tableofcontents


% 06.10.2016, 23:55, as PDF to beat.stettler@ins.hsr.ch

\section{Aufgabenstellung}
\subsection{Einleitung}
Gemäss der Aufgabenstellung

\begin{quotation}
	The fictional company BetaHouse Inc. is specialized in the high-frequency trading sector. In this exercise you will design the LAN part of the network, based on customer specifications. Part two will focus on the WAN integration with several remote locations throughout Switzerland (10+).
	
	Start with analyzing and characterizing the desired goals. Then go ahead with a logical network design. Based on that define the physical network. Document, review and change if needed until you get a satisfying solution. 
	
	\textbf{It’s important to describe your design decisions in your documentation.}
\end{quotation}

\subsection{Angaben zur Firma}

\begin{description}
	\item[Firmenname] BetaHouse Inc.
	\item[Niederlassungen] 10+, verbunden über WAN
\end{description}

%TODO Michi: Include the cool diagram

\subsection{Vorgehen}
\begin{enumerate}
	\item Analyse der Kundenansprüche
	\item Designentscheidungen ausformulieren
	\item Aufbau des Campus LAN Design
	\item Aufbau des Adressierungs- \& Namensschema
	\item Analysieren Nutzung von Cloud Diensten
\end{enumerate}

\section{LAN Design Headquarter}

\subsection{Kundenansprüche an das Netzwerk}

Availability / Redundancy requirements

Major traffic flows

Application QoS requirements

expected growth in the comming 5 years -> 20%


\subsection{Campus LAN Design}

\subsubsection{Logische Netzwerkkarte}
%MUST cover WAN access
%MUST cover Internet access
%MUST cover reasoning for design
\subsubsection{Physische Netzwerkkarte}
%MUST cover reasoning for design

\subsubsection{Adressierungs- \& Namensschema}
\paragraph{IPv4} 
\begin{table}[h]
	\centering
	\begin{tabu} to \linewidth {l l l}
		\toprule 
		Subnetz & VLAN & Beschreibung \\
		\midrule
		
		\bottomrule 
	\end{tabu} 
	\caption{IPv4 Addressing Scheme}
\end{table}

\paragraph{IPv6}
\begin{table}[h]
	\centering
	\begin{tabu} to \linewidth {l l l}
		\toprule 
		Subnetz & VLAN & Beschreibung \\
		\midrule
	
		\bottomrule 
	\end{tabu} 
	\caption{IPv6 Addressing Scheme}
\end{table}


\subsection{Technologie \& Protokolle}
\subsubsection{Layer 2}
Wir verzichten auf STP, da wir für alle Bereiche eigene IP Subnetze verwenden. 

\subsubsection{Layer 3}
Für das Routing wird OSPF verwendet. 

\subsection{Benötigte Komponenten}
\subsubsection{Access Ports}
\begin{table}[h]
	\centering
	\begin{tabu} to \linewidth {l l l l l}
		\toprule 
		Location & Abteilung & Anz. Mitarbeiter & Anz. Drucker & Benötigte Access Ports \\
		\midrule
		HQ 1. Stock & Banking und Gäste & 150 & 15 & $150 + 15 + 1 = 166$ \\ 
		HQ 2. Stock & Banking & 150 & 15 & $150 + 15 = 165$ \\
		HQ 3. Stock & Trading & 30 & 3 & $2 \cdot 30 + 3 = 63$ \\ 
		HQ 4. Stock & IT und HR & $10 + 20 = 30$ & 3 & $30 + 3 = 33$\\
		Branch  & Banking & 20 & 2 & $20 + 2 = 22$\\
		\bottomrule 
	\end{tabu} 
	\label{tbl:require_access_ports}
	\caption{Benötigte Access Ports}
\end{table}

\subsection{Switches}
Wir setzen im gesamten Netzwerk Cisco Catalyst 3750-48TS mit je 48 Ports ein. Diese bieten die Möglichkeit die L3-Switches als Stack zu organisieren. Jeder Stack wird über einen eigenen logischen Hostnamen angesprochen. Um die hohe Verfügbarkeit für die Trading Abteilung zu garantieren, schliessen wir die VoIP Phone an einem separaten Access Port an. Die 30 Mitarbeiter werden zu jede 15 Personen an einen L3 Switch angeschlossen. So bleiben bei einem Ausfall immer die Hälte der Trader voll operationsfähig. Um einen Ausfall möglichst zu vermeiden, sind alle Switches an einer USV angeschlossen. Da bei einem Ausfall der Switches in der Trading Abteilung grosse Kosten entstehen, hält sich die IT einen vorkonfigurierten Backup Switch in Ihrem Stockwerk. Damit kann im Notfall der Switch ohne grössere Zeitverluste ausgetauscht werden. Alle weiteren Abteilungen haben eine normale Wichtigkeit und werden gemäss ihrer Mitarbeiterzahl angeschlossen. Es wurde aber auch bei Ihnen ein Wachstum eingerechnet. Aus der Tabelle \ref{tbl:require_access_ports} folgt, dass folgende Switches angeschafft werden müssen. 
\begin{table}[h]
	\centering
	\begin{tabu} to \linewidth {l l l X X}
		\toprule 
		Bereich & Anzahl & Komponente & Hostname & Bemerkung \\
		\midrule
		Guest WLAN & 1x & IEEE 802.11ac Access Point & HQ\_AP\_1 & \\
		Banking HQ 1. floor & 4x & 48 Port Catalyst 3750-48TS & HQ\_AS\_1 & \\
		Banking HQ 2. floor & 4x & 48 Port Catalyst 3750-48TS & HQ\_AS\_2 & \\
		Trading HQ 3. floor & 2x & 48 Port Catalyst 3750-48TS & HQ\_AS\_3 & \\
		IT und HR HQ 4. floor & 1x & 48 Port Catalyst 3750-48TS & HQ\_AS\_4 & \\
		Branch 1 & 1x & 48 Port Catalyst 3750-48TS & BR1\_AS\_1 & \\
		Branch 2 & 1x & 48 Port Catalyst 3750-48TS & BR2\_AS\_2 & \\
		Branch 3 & 1x & 48 Port Catalyst 3750-48TS & BR3\_AS\_3 & \\
		\bottomrule 
	\end{tabu} 
	\caption{Benötigte Switches}
\end{table}

\section{WAN Design}

\subsection{Kundenansprüche an das Netzwerk}

\subsection{WAN Network Design}
%MUST cover remote locations, external services like web, mail
%MUST include security considerations

\subsubsection{Adressierungs- \& Namensschema}

\subsubsection{Logische Netzwerkkarte}
%MUST cover HQ and 3 Branch Offices
\subsubsection{Physische Netzwerkkarte}

\subsection{Technologie \& Protokolle}
\subsubsection{Layer 2}
\subsubsection{Layer 3}

\subsection{Benötigte Komponenten}

\section{Organisation IT-Applikationen}

\subsection{Positionierung Applikationsserver}
%Where	would	you	place	servers	in	your	network?	Decide	for	each	of	the	following	applications,	where	to	place	them	and	why?	Domain-Control	Servers,	Mainframe	for	Banking-Application,	High-Frequency-Trading-Server,	VoIP	Server,	Backup	Server

Um eine ausfallsichere und schnelle Applikations-Infrastruktur zu ermöglichen, empfiehlt es sich, die Applikationsserver folgendermassen zu verteilen.

\subsection{Cloud Dienste}
%Describe	possible	limitations	with	this	design	if	the	company	would	be	starting	to	use	cloud	services.	Describe	possible	limitations	with	this	design	if	the	company	would	be	starting	to	use	cloud	services.	s


\appendix

\section{Schemes}
% TODO physical topology (map of the network, datacenter outlet)
% TODO logical topology (map of the network, building block view, layer2 map(vlan), layer 3 map (ospf, areas??) )

\subsection{Datacenter Design}
Das Schema ist im Anhang \ref{appendix:datacenter} zu finden.

\subsection{3 Tier des Application Layers}
Das Schema ist im Anhang \ref{appendix:3_tier_application_layer_physical} zu finden.



% List of tables
\listoftables


\label{appendix:datacenter}
\includepdf[pages={1},landscape=true]{appendix/schemes/datacenter.pdf}

\label{appendix:3_tier_application_layer_physical}
\includepdf[pages={1}]{appendix/schemes/3_tier_application.pdf}


% Code Listings
% \lstlistoflistings

% List of figures
% \listoffigures

% Bibliography
% \bibliographystyle{plain} 
% \bibliography{literatur}


\end{document}

