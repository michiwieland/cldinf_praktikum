\input{../template.tex}

% Dokumentinformationen
\newcommand{\SUBJECT}{Report}
\newcommand{\TITLE}{Cloud Infrastructre Lab 7}

\begin{document}
	
% Front page
\title{\TITLE}
\subject{\SUBJECT}
\author{\SECONDAUTHOR}
\author{\AUTHOR}
\affil{\INSTITUTE}
\date{\today}
\maketitle

% Table of contents
\tableofcontents


\section{Custom Clos Topology}
\subsection{Requirements}
Auf der vorgegebenen VM müssen noch folgende Pakete installiert werden
\begin{lstlisting}[language=bash]
# requirements
sudo apt-get install python-pip
sudo pip install -r ./requirements.txt

# call
./clos-nnetwork <spines> <leaves>
\end{lstlisting}

\subsection{Mininet CLI}
\begin{lstlisting}[language=bash]
# clean mininet
sudo mn -c

# debug
sudo mn -v debug

# print configuration
mininet> pingall
mininet> nodes
mininet> net
mininet> dump

# host commands
mininet> h1 ifconfig -a

# switch commands
mininet> s1 ifconfig -a

# connectivity test
mininet> h1 ping h2

# pyhton interpreter
mininet> py 'hello ' + 'world'
\end{lstlisting}

\subsection{Source Files}
\subsubsection{requirements.txt}
Wir verwenden für die Parameter Übergabe die CLI Description Language von Docopts. 
\lstinputlisting{sourcen/requirements.txt}

\subsubsection{clos-network.py}
Das Clos Python Skript nimmt zwei Parameter entgegen und erstellt die gewünschte Anzahl Leafs und Spines. Die Leaf Knoten werden mit allen vorhandenen Spine Knoten verknüpft. Ebenfalls wird an jeden Leaf Knoten genau ein Host angehängt.
\lstinputlisting{sourcen/clos-network.py}

\subsubsection{stpswitch.py}
Damit die Pings funktionieren, dürfen in dem Netzwerk keine Loops existieren. Weder der standardmässig verwendetete ovs-controller, noch NOX's pyswitch, noch POX l2\_learning switch unterbinden Loops als Standardeinstellung. Für ein Cloud Netzwerk würde sich ein OpenFlow Multipath Controller anbieten, einfachheitshabler haben wir das Problem aber mit STP gelöst. 
\lstinputlisting{sourcen/stpswitch.py}

\subsection{Tests}
Nachdem der STP Tree konvergiert hat, können sämtliche Hosts in der Clos Topology gepingt werden.
\begin{figure}[h]
\centering
\includegraphics[width=0.7\linewidth]{images/ping_result}
\caption{pingall Resultat}
\label{fig:pingresult}
\end{figure}

\clearpage

\subsection{Topologie}
\begin{figure}[h]
	\centering
	\includegraphics[width=0.9\linewidth]{images/scheme}
	\caption{Clos Topologie mit 2 Spines und 4 Leafs}
	\label{fig:scheme}
\end{figure}




\appendix

\section{Schemes}
% TODO physical topology (map of the network, datacenter outlet)
% TODO logical topology (map of the network, building block view, layer2 map(vlan), layer 3 map (ospf, areas??) )

\subsection{Datacenter Design}
Das Schema ist im Anhang \ref{appendix:datacenter} zu finden.

\subsection{3 Tier des Application Layers}
Das Schema ist im Anhang \ref{appendix:3_tier_application_layer_physical} zu finden.



% List of tables
\listoftables


\label{appendix:datacenter}
\includepdf[pages={1},landscape=true]{appendix/schemes/datacenter.pdf}

\label{appendix:3_tier_application_layer_physical}
\includepdf[pages={1}]{appendix/schemes/3_tier_application.pdf}


% Code Listings
% \lstlistoflistings

% List of figures
% \listoffigures

% Bibliography
% \bibliographystyle{plain} 
% \bibliography{literatur}


\end{document}

